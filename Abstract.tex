\documentclass[11pt]{article}


\usepackage{amsmath}
\usepackage{lipsum} % This package is just to generate text for the example

\usepackage{geometry}
\geometry{
	left=1in,  % Left margin
	right=1in, % Right margin
	top=1in,     % Top margin
	bottom=1in   % Bottom margin
}

\title{Retractions in Management Scholarship: Speaking to the Elephant in the Room}
\author{Karthikeyan Balakumar, Parijat Lanke, Advaita Rajendra}
\date{}

\begin{document}
	
	\maketitle
	
\section*{Abstract}
		To make mistakes is to be human, but certain inevitable mistakes have an impact that might persist, even post-repair. One such mistake in academia is publishing what shouldn’t have been published. The process of ‘repairing’ that irreversible ‘damage’ done is termed a retraction. A retraction in academic publishing refers to formally withdrawing or removing a previously published article from the scholarly literature. Typically, retractions are issued by journals or publishers along with a formal notice that explains the reason for the retraction. These notices alert the broader academic community and seek to educate them about refining ongoing work in case they are based on past retracted work. Despite this, it is highly probable that several retracted papers find their way into ‘reliable’ scholarly literature. Yet, without retractions, maintaining the integrity and credibility of the scientific literature would be challenging (Bakker et al., 2022), and Issues such as scientific misconduct, research errors, and the inability to reproduce results may never see the light of day (Wager \& Williams, 2011; Redman et al., 2008). They provide a means to address issues such as plagiarism, duplicate publications, and unreliable data, thereby upholding the quality of scientific research (Nair et al., 2019; Shi et al., 2020).
		
		Our study has twin objectives. First, our study understands the pasture of retracted papers in management scholarship. Second, we dwell deeper into the meanings that these retractions hold for the progress of management scholarship through the lenses of speech act theory and the idea of normative power. For the first part of the study, we utilise data retrieved from Retraction Watch and analyse it using social network analysis and bibliometric techniques. For the second objective, we reflect on the results through theoretical lenses. 
		
		Our research seeks to contribute to the ongoing discourse on the prevalence of retractions in scholarship and offer insights from a policy standpoint. Limitations of our research and future research directions are also discussed.
	
	
\section*{Keywords}
	Retraction; Management Scholarship; Social Network Analysis; Bibliometrics; Normative Power

	
\end{document}
